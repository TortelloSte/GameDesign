\documentclass{report}
\usepackage[italian]{babel}
\usepackage[a4paper,top=2cm,bottom=2cm,left=3cm,right=3cm,marginparwidth=1.75cm]{geometry}
\usepackage{amsmath}
\usepackage{graphicx}
\usepackage[colorlinks=true, allcolors=blue]{hyperref}

\title{Documento di Game Design}
\author{Nome gioco}
\begin{document}
\maketitle
% queste prime righe servono per andare a personalizzare lo stile del documento e il nome del gioco.

\paragraph{Introduzione} .\par
Inserire in questa sezione una breve introduzione del gioco che si vuole andare a sviluppare


\section{Concept}

\subsection{Elevator pitch}

esempio:

Stiamo sviluppando un gioco di ruolo open-world ambientato in un mondo fantasy medievale ispirato al folklore europeo. Il suo gameplay è fortemente basato sulla trama e basato sul libro più venduto "Il conte degli anelli". È "The Witcher" che incontra "Dark Souls" che incontra "Il Trono di Spade".

\subsection{Overview}
\begin{itemize}
  \item Genere: 
  \item Target
  \begin{itemize}
    \item età
    \item genere
  \end{itemize}
  \item Monetizzazione
  \item Piattaforme e requisiti di sistema
\end{itemize}

\subsection{Theme e Setting}

 ... è un gioco di ruolo ambientato in una versione fantasy storica dell'Europa medievale. Il giocatore assume il ruolo del Prescelto ed esplora un mondo aperto di 450 chilometri quadrati. Sebbene l'ambientazione geografica del gioco sia storicamente accurata , il gioco presenta molti elementi soprannaturali.

\section{Game setting}
\subsection{Locations}
\begin{itemize}
\item nome luogo
      \begin{itemize}
        \item descrizione generale

        descrivere qui la zona di gioco nel dettaglio
        
        \item ispirazione (inserire link di foto o video come esempio)
        \item avvenimenti nella zona di gioco
        \item connessioni con le altre aree di gioco
      \end{itemize}
\end{itemize}
\subsection{Storia}

Inserire dentro questa sezione la storia del gioco
\subsection{Personaggi}
\begin{itemize}
    \item Protagonista
        \begin{itemize}
            \item Backstory
            \item personalità
            \item aspetto
            \item abilità speciali
        \end{itemize}
    \item Nemico Base
    \item Boss
    \item NPCs
\end{itemize}

\section{Gameplay e meccaniche di gioco}
\begin{itemize}
    \item Gameplay
    \begin{itemize}
        \item Preogressione del gioco
        \item obiettivi di gioco
    \end{itemize}
    \item Meccaniche
    \begin{itemize}
        \item Fisica di gioco
        \begin{itemize}
            \item Movimenti
            \item Oggetti
            \item Azioni
            \begin{itemize}
                \item combattimento
                \item interazione con gli oggetti
                \item parlato
                \item letto
                \item ecc...
            \end{itemize}
        \end{itemize}
    \end{itemize}
    \item Cheats e easter egg
\end{itemize}

\section{Livelli}
\subsection{Livello 1 - Prologo}
\begin{itemize}
    \item sinossi
    \item obiettivi
    \item location
    \item procedura del livello
\end{itemize}

\subsection{Livello 2 - Inizio}
\subsection{Livello 3 - Evento x}
\subsection{Livello n - Avvenimento N}

\section{Asset}
\subsection{Grafica}
\begin{itemize}
    \item Consept Art

    inserire dettagliatamente per ogni location inserita, personaggio inserito come vuole essere sviluppato
    \item Modelli e texture
\end{itemize}
\subsection{Audio}
\begin{itemize}
    \item Effetti sonori
    \item Musica
    \item Voci
\end{itemize}
\end{document}