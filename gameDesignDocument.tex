\documentclass{report}
\usepackage[italian]{babel}
\usepackage[a4paper,top=2cm,bottom=2cm,left=3cm,right=3cm,marginparwidth=1.75cm]{geometry}
\usepackage[colorlinks=true, allcolors=blue]{hyperref}

\title{Documento di Game Design}
\author{Nome del Gioco}

\begin{document}
\maketitle

\tableofcontents
\section{Introduzione}
All'interno di questa sezione viene indicata l'introduzione al gioco e al suo mondo che si vuole andare a sviluppare.
In questa parte di documento bisogna introdurre quale sarà la panoramica del gioco, cosi da far comprendere al lettore o i lettori del seguente documento quello che si andrà a sviluppare senza entrare troppo nel dettaglio, in quanto sono presenti le sezioni dedicate per tutto lo sviluppo.
L'introduzione deve essere breve e concisa, di norma non deve superare le cinque righe di testo, o le 500 parole.

\newpage
\section{Concept}

Questa sezione sottostante serve per andare a spiegare al lettore del documento quello che è lo scopo del gioco, quindi gli obiettivi e come vuole essere sviluppato, fare una mappatura per far comprendere quale è principalmente il target del gioco, e soprattutto spiegare il genere del gioco che vuole essere sviluppato, inoltre si va a trattare anche dell'aspetto di marketing per capire come il gioco possa essere utilizzato per monetizzare e per le piattaforme di riferimento.

All'interno della sezione di Concept inoltre viene inserito il theme e setting, ossia l'area che spiega il tema del gioco, anche la sua morale, e il settaggio quindi l'ambientazione fisica di dove si svolge la storia del gioco e dove il giocatore si troverà a giocare in questo mondo

\subsection{Elevator Pitch}
L'elevator Pitch indica una sezione e porzione di documento che deve andare a descrivere gli obiettivi del documento, per esempio andare a descrivere che cosa si sta sviluppando.
Un esempio per questa sezione può essere: 

Lo sviluppo del gioco è incentrato su un'esperienza di avventura nel mondo dei fantasmi, con un tocco di fantasy medievale, ispirato a ... 
Lo stile del gameplay deve essere come: esempio di gioco 1, esempio 2 ecc.

\subsection{Overview}

Questa sezione deve spiegare nel dettaglio le scelte intraprese per la scelta del genere del gioco con la sua completa descrizione, il target di utenti a cui si punta, come si intende monetizzare con il gioco, e le piattaforme sul quale vuole essere fatto, e infine i requisiti di sistema che devono essere sia il minimo che il massimo per andare a far comprendere con che risoluzione/materiale si potrà andare a sviluppare
\begin{itemize}
  \item Genere: Il gioco che si intende sviluppare sarà uno (Sparatutto) -> descrizione completa di cosa è un gioco sparatutto cosi da far comprendere che cosa comprende questa scelta di genere
  \item Target: In questa sezione va inserito il target di riferimento del gioco quindi come descritto nell'elenco sottostante si tratta del genere di giocatore, quindi di norma va compilato con tutti* e l'età, quindi una prima indicazione per la certifica del PEGI se necessaria, o comunque per far comprendere agli sviluppatori e agli story teller quello che è il range di età e quindi quanto possono essere espliciti nello sviluppo del videogioco.
  \begin{itemize}
    \item Età
    \item Genere
  \end{itemize}
  \item Monetizzazione:
  In questa sezione viene inserito il metodo con cui si vuole monetizzare e il piano di marketing che viene inserito all'interno del gioco, questo significa il costo di produzione, il valore economico del gioco al suo rilascio, se ci sono acquisti in game inserire anche quelli nel dettaglio. Inoltre va inserito anche il costo di pubblicazione e per le pubblicità del gioco, cosi da comprendere quali sono i costi che dovranno essere affrontati per lo sviluppo del gioco
  \item Piattaforme e requisiti di sistema:
  In questa sezione vanno inserite le piattaforme sulle quali il gioco deve essere sviluppato, che sono differenti sia per codice, che per meccaniche di gioco, quindi capire preventivamente se il gioco sarà solo per PC o se si vuole portare anche su altre console.

  Inoltre vanno inseriti i requisiti di sistema per il download, quindi la memoria da utilizzare, peso degli aggiornamenti, se per PC i requisiti grafici, audio ecc...
  I requisiti devono essere minimi e massimi di sistema un esempio può essere:
  MINIMI:
    Richiede un processore e un sistema operativo a 64 bit
    Sistema operativo *: Windows 7 64-bit, Service Pack 1
    Processore: Intel Core i5-2300 2.8 GHz / AMD FX-6300, 3.5 GHz
    Memoria: 6 GB di RAM
    Scheda video: GeForce GTX 460, 1 GB / Radeon HD 6870, 1 GB
    DirectX: Versione 11
    Memoria: 8 GB di spazio disponibile
    Scheda audio: DirectX 11 sound device
    Note aggiuntive: Low Settings, 60 FPS @ 1080p
  MASSIMI:
  Richiede un processore e un sistema operativo a 64 bit
    Sistema operativo: Windows 10 64-bit
    Processore: Intel Core i5-4570 3.2 GHz / AMD FX-8350 4.2 GHz
    Memoria: 8 GB di RAM
    Scheda video: GeForce GTX 660, 2 GB / Radeon HD 7870, 2 GB
    DirectX: Versione 11
    Memoria: 8 GB di spazio disponibile
    Scheda audio: DirectX 11 sound device
    Note aggiuntive: High Settings, 60 FPS @ 1080p
\end{itemize}

\subsection{Theme e Setting}
Per terminare l'area del documento dedicata al Concept e gli attori generali che devono andare a sviluppare il gioco stessi si deve andare a delineare una linea guida per sviluppatori e grafici per far comprendere al meglio come deve essere strutturato il gioco, quindi in questa area va inserita l'ambientazione generica del gioco, per esempio spiegare che il gioco avrà una ambientazione fantasy con tratti storici, indicare quale sarà il ruolo del giocatore dentro questo ambiente, definirne la grandezza e se presente per il gioco sarebbe il caso di definire in questa area di gioco la morale che vuole essere insegnata attraverso il gioco stesso.
Un esempio potrebbe essere: Il gioco in questione viene ambientato in una versione futuristica fantasy di una New York di un altro pianeta. Il progragonista, e quindi il giocatore che ne assume il ruolo deve essere pronto a vestire i panni dell'investigatore per andare a far luce sui vari misteri di questo nuovo mondo, di circa 1000 chilometri quadrati con elementi sovrannaturali e futuristici.

\newpage

\section{Game Setting}
Questa sezione sarà quella più lunga e corposa di tutta il documento perché servirà a descrivere il gioco in ogni sua parte e in ogni dettaglio che lo rappresenta, questa sezione risulterà dunque fondamentale a tutti i team di sviluppo e grafica per andare a comprendere quello che è il gioco e cosa va sviluppato, quindi fa da linea guida a tutto lo sviluppo del documento. In questa sezione è indicato aggiungere materiale visivo per mostrare esempi di stile grafico o di gameplay da sviluppare.

\subsection{Locations}
All'interno di questa sezione di documento dovranno essere inserite tutte le aree giocabili del gioco. Se nel gioco sono presenti differenti biomi bisognerà descrivere nella sezione dedicata il bioma e successivamente ogni area di quel Bioma. Questa area serve per andare a descrivere graficamente come deve essere sviluppata l'area di gioco. Oltre ad una descrizione generale dell'area e del luogo è comodo inserire i link e foto per dare ispirazione ai grafici, soprattutto per andare a definire quello che è lo stile grafico ed artistico.
Buona norma è quella di descrivere quelli che sono gli avvenimenti che accadono all'interno di questa area, o gli incontri con NPCs. Infine vanno inserite tutte le connessioni con le altre aree di gioco.

\begin{itemize}
\item \textbf{Bioma 1 : esempio di Macro area}

Descrizione intera di questa macro area, che possa andare a narrare graficamente quello che deve essere questa area di gioco, quali sono gli scopi di questa area.
    \begin{itemize}
    \item \textbf{Luogo 1: esempio Piana di partenza}    
        \begin{itemize}
            \item Descrizione Generale: All'interno di questa area di gioco viene descritta quella che è l'intera area di gioco, in questo caso deve essere presente una descrizione di quella che graficamente è l'area, il suo scopo e i personaggi che sono al suo interno.
            \item Ispirazione: All'interno di questa area vanno inseriti i link a foto o video come esempio, cosi da poter mostrare graficamente quello che è lo stile artistico, o che dia ispirazione per andare a costruire l'area definita.
            \item Avvenimenti nella Zona di Gioco: Questa sezione deve indicare tutte le missioni, avvenimenti di gioco che accadono in questa area, come incontri, scontri, avvenimenti storici, e task da raggiungere in questa area di gioco.
            \item Connessioni con Altre Aree di Gioco: In questa area di gioco vanno inserite le connessioni tra le aree. Per esempio: La piana di partenza si connette a Nord con la landa degli assassini e a Ovest con l'ingresso della locanda di Bob.
        \end{itemize}
    \end{itemize}
\end{itemize}

\subsection{Storia}

Questa sezione riveste un ruolo assolutamente \textbf{essenziale}. È imperativo che questa parte del progetto sia completamente dettagliata con l'intera trama del gioco, al fine di garantire una comprensione approfondita a tutti coloro che partecipano al progetto. La chiave è mantenere un filo conduttore chiaro e assicurarsi che gli elementi descritti in questa sezione siano approfonditamente delineati, è inoltre importante controllare che quello che viene descritto in questa area del documento sia presente anche in tutte le altre zone definite nel documento.

\subsection{Personaggi}

Questa area del documento va a descrivere tutti i personaggi che vengono presentati all'interno del gioco, ogni personaggio, anche quelli minori, vanno descritti con una Backstory, la sua personalità (la quale permette di comprendere come dovrà essere disegnato psicologicamente e anche gli atteggiamenti del parlato.), l'aspetto è essenziale per andare a comprendere come verrà successivamente disegnato il personaggio, e nel caso in cui ne possedesse le abilità speciali in combattimento e non.

Un esempio:
\begin{itemize}
    \item Protagonista: il protagonista del gioco ha sicuramente un nome da inserire in questa sezione
        \begin{itemize}
            \item Backstory: Ogni personaggio ha una sua storia personale che magari non è descritta nella sezione Storia del documento, quindi in questa area si può indicare tutta la storia del personaggio.
            \item Personalità: Ogni personaggio ha la sua personalità psicologica, quindi c'è da capire se questo personaggio è un ragazzo adolescente ribelle, o un rigido militare inflessibile ecc.
            \item Aspetto: descrivere fisicamente il personaggio che si sta realizzando, per esempio: Il protagonista è un uomo adulto, ex militare di 1,92m che presenta un grande tatuaggio sul volto a forma di sole, capelli rossi, e un occhio vitreo... ecc
            \item Abilità Speciali: Riesce a sparare un colpo dal suo occhio vitreo, ricarica automatica che da +1 stamina ecc...
        \end{itemize}
    \item Nemico Base

    Per ogni personaggio presente all'interno del documento vanno ricompilate tutte le sezioni elencate precedentemente.
    \begin{itemize}
        \item Backstory
        \item Personalità
        \item Aspetto
        \item Abilità speciali
    \end{itemize}
    \item Boss del gioco
    \item Boss Segreto
    \item NPCs
    \item NPCs senza ruolo
\end{itemize}

\newpage

\section{Gameplay e Meccaniche di Gioco}

Questa sezione del documento aiuta soprattutto i team di sviluppo per andare a capire come deve essere sviluppato il gioco lato codice, cosi da andare a comprendere tutte le meccaniche di gioco che verranno implementate
\begin{itemize}
    \item Gameplay
    \begin{itemize}
        \item Progressione del Gioco

        Indicare in questa sezione quelli che sono gli step fondamentali all'interno del gioco cosi da poter delineare una linea guida all'interno del gioco, la partenza, primo avvenimento, secondo avvenimento, ipotetica strada uno o due ecc...
        \item Obiettivi di Gioco

        Indicare quelli che sono gli obbiettivi di gioco, gli step necessari per andare a completare il gioco.
    \end{itemize}
    \item Meccaniche
    \begin{itemize}
        \item Fisica di Gioco
        \begin{itemize}
            \item Movimenti

            fare una lista completa di quelli che sono i movimenti da implementare, e da sviluppare, andando a specificare per che cosa sono pensati, personaggio, ambiente
            \item Oggetti

            definire quelli che sono gli oggetti presenti all'interno del gioco, gli interagibili e i non, e quelli che sono scenici ecc...
            Definire quelli che sono gli aspetti fisici degli oggetti per andare a capire come farli interagire con i personaggi. 
            \item Azioni

            Esistono differenti tipi di Azioni che possono essere sviluppate all'interno di un videogioco, queste vanno pensate e strutturate e va implementata ad ognuna di esse una fisica particolare, infatti i salti, gli scatti ecc hanno una precisa fisica che emula quella reale che va emulata e riportata in quella del gioco.
            \begin{itemize}
                \item Combattimento
                \item Interazione con gli Oggetti
                \item Dialoghi
                \item Lettura
                \item E altro...
            \end{itemize}
        \end{itemize}
    \end{itemize}
    \item Cheats e Easter Egg

    Questa area non è obbligatoria ma, potrebbe essere essenziale per rendere maggiormente divertente e carino il gioco, infatti aggiungere easter egg rappresenta una grande stima rivolta agli altri giochi, serie o qualunque cosa venga citato in maniera banale o meno. Inoltre si possono inserire all'interno del gioco elementi che permettono di saltare boss fight, aree intere ecc, andando a nasconderli in bella vista.
\end{itemize}

\newpage

\section{Livelli}

Molti giochi vengono suddivisi in livelli, la seguente è una struttura che possa descrivere il susseguirsi dei vari livelli, e di come devono essere strutturati.

\subsection{Livello 1 - Prologo}
\begin{itemize}
    \item Sinossi: Narrazione di quello che è il livello e la sua narrazione e descrizione
    \item Obiettivi: Scopo del livello per andare terminarlo
    \item Location: Luogo del livello dove viene narrato cosi da poterlo inserire in un determinato ambiente
    \item Procedura del Livello: come deve essere svolto il livello, quindi 
\end{itemize}

\subsection{Livello 2 - Inizio}
\subsection{Livello 3 - Evento X}
\subsection{Livello N - Avvenimento N}

\newpage

\section{Asset}
\subsection{Grafica}
\begin{itemize}
    \item Concept Art: Dettagli per ogni location e personaggio.
    \item Modelli e Texture
\end{itemize}

\subsection{Audio}
\begin{itemize}
    \item Effetti Sonori
    \item Musica
    \item Voci
\end{itemize}

\end{document}
